% \documentclaa[]{}

\begin{document}

\title{Private Collaborative Editing of Documents Over the Internet}

\authorinfo{Benjamin Ylvisaker}

\maketitle

\begin{abstract}

Applications that give teams of people the ability to collaboratively edit documents over the Internet have become popular.
Examples include Google Docs, Evernote and Dropbox.
With few exceptions, these applications give the providers of the service complete access to the teams' data.
This raises multiple serious privacy concerns.

In response to related privacy concerns, end-to-end encryption has been encorporated into many Internet applications.
End-to-end encryption has been most successful with applications in which either users manage non-shared data (for example, password managers) or messaging applications with only transient, not persistent, state.

This project is the first to combine end-to-end encryption with shared, mutable state.
The key technical challenge is that the clients that belong to a team must perform the consistency protocol amongst themselves, with no central authority.
We assume that each team member will only be actively connected to the network intermittently, but also has a highly available read-only copy of the team's data.

\end{abstract}

\section{Introduction}

Privacy concerns have become important to many Internet users in recent years.
It is known that many adversaries, including powerful government and corporate organizations, thieves and vandals, have violated huge numbers of users' privacy in myriad ways.
One important tool in the struggle for privacy on the Internet is encryption, and in particular end-to-end encrypted applications.

End-to-end encryption has become common (though not universal) in applications like password management and internet messaging.
The key benefit of end-to-end encryption is that all service providers (both network and storage) only see the encrypted data.
Without breaking the encryption, lying about the application's behavior, or otherwise maliciously circumventing the system, the system provider cannot get the user's unencrypted data.
This is useful for several reasons.

Unfortunately there is a big gap in the range of applications that can currently be end-to-end encrypted.
As long as the system provider is only providing passive storage (as in password managers) or transient communication (as in messaging applications), the overall architecture is well understood.
However, in applications where a team of users collaboratively edit a document or database, the service provider also runs the consistency protocol on their servers.
In order to do this, the service provider needs access to the unecrypted data.

One gap in the end-to-end encryption landscape is applications that allow teams of users to collaboratively edit a document/database.

\section{Security and Attack Models}

The goal of the ManyHands protocol is to allow a team of users to collaboratively edit data online with as little support from any central service provider.
The rationale for the service provider restriction is to keep the attack surface as small as possible.

\section{Architecture}

The ManyHands protocol involves a central server, a cloud storage service for each user, and a client computer for each user.
The central server plays a small and inessential role; it is merely a convenience to help users log in from anywhere on the Internet.

We assume that the cloud storage services are highly available.
The access policy required of the storage service is trivial.
The user must be able to read and write anything in her own account's storage.
All other entities are allowed to read anything in the user's cloud storage account, but are not allowed to write anything.
World-readability is not a privacy issue, because all sensitive data is encrypted with keys that only team members have access to.

The single-writer policy is somewhat interesting.
Allowing more than one user to write to a single cloud storage location implies that the cloud storage provider has some information about the 

\end{document}

Alvarado Dodds Kuenning Libeskind-Hadas
