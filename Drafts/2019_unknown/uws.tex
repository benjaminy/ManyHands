\documentclass{article}

\begin{document}
\title{Privacy-Preserving Collaborative Editing for the Masses}
\author{Benjamin Ylvisaker}

\maketitle

\section{Introduction}

Collaborative editing applications are useful tools for helping teams work and socialize together.
Popular examples include Google Docs, Evernote, Etherpad, Slack and Trello.
In all widely used collaborative editing (CE) applications, a central service integrates all teammates' edits into a consistent database/document and stores that database.
In many cases this service is run by the same organization that develops the application.
If a team has concerns (even mild ones) about protecting their data from exposure or tampering, this architecture is extremely problematic.
The team's data is vulnerable to mishandling and attacks from the provider themselves, rogue employees, malicious hackers, governments/courts, etc.

In recent years, similar privacy concerns have led to wide adoption of secure messaging protocols and applications like OTR, Signal, WhatsApp and Telegram.
CE is similar to messaging, but with two additional challenges:
\begin{enumerate}
\item The data is persistent.
  Edits in CE systems are simlar to messages in messaging systems, but basic functionality requires that ``old'' edits be accessible to clients in an efficiently queryable format.
\item Edits from teammates need to be integrated into a consistent document{\slash}database.
  Concurrent (potentially conflicting) edits are a challenge that needs to be addressed in some way.
\end{enumerate}

There are a few experimental secure/private CE architectures (e.g. SPORC, TRVE Data, PeerPad), but to the best of our knowledge no applications developed on such platforms have gained a significant user base.
The focus of this project is identifying and mitigating remaining impediments to wide adoption of secure CE technologies.
By way of analogy, we note that secure messaging technologies (e.g. PGP/GPG) existed for many years before the current generation of secure messaging applications, but their use was mostly confined to a small group of people with serious data privacy concerns.
By making secure messaging almost as usable/convenient as less secure systems, the new generation achieved mass adoption (aided, no doubt, by increasing public concern about data security and privacy).
We are developing a protocol called United We Stand (UWS) to explore whether secure collaborative editing can similarly be made as usable as traditional centralized architectures.

\section{Security-Convenience-Cost Trade-offs}

There are trade-offs between security/privacy and convenience/usability.
Clever engineering can sometimes bring benefits in one dimension with little loss in the other, but it is unlikely that it is possible to achieve maximal security and convenience in the same system.

Our intuition for navigating these trade-offs in UWS involves recognizing the spectrum of attacks from mass surveillance and scattershot scripted attacks at one end to advanced and individually/organizationally targeted security and privacy attacks at the other.
We intend to provide strong protection against untargeted attacks, but where there is tension between convenience and protection from targeted attacks, we prefer not to compromise on convenience.
This is related to our primary motivation of creating a CE framework that can compete with conventional centralized systems in terms of usability.
In other words, our goal is to build a system that provides good privacy protection to many people, rather than iron-clad protection to the few who are willing to compromise in other areas.

In addition to these security-convenience trade-offs, economic questions play an important role.
In particular, many providers of conventional CE applications derive profit from users' data.
This can be as simple as selling users' data to advertisers, or more complicated crap like building models for personalizing other services.
Secure CE systems generally aim to prevent anyone outside of teams themselves from gaining access to the team's data (ours certainly does).
This means that we have to think creatively about who pays for the storage, communication and computation resources involved.
We generally prefer to put these costs on end users.
One important hole in the economic story that we leave for cleverer minds is funding the development and maintenance of application code.

% This is already a challenge in the messaging context; for example,   (who pays for Signal?)

\section{Design Principles for Usable/Convenient Secure CE}

\begin{itemize}
\item Teammates cannot be assumed to be online simultaneously.
  Therefore, some kind of server is necessary, but servers are dangerous, so their role should be minimized.
\item Smooth offline operation is necessary.
\item ``Remote access'' (better jargon for this?)
\item Low latency is good, but moderate or high latency is tolerable for some applications.
\item Scalability is good, but ...
\end{itemize}

\subsection{Servers}

Purely peer-to-peer systems (like PeerPad) cannot match the usability of conventional CE systems, because they rely on teammates being simultaneously online.
(P2P systems do not need to rely on \emph{all} teammates being online simultaneously, but rather propagate edits from user to user.)
We consider requiring such live P2P connections to be an unacceptable usability downgrade relative to centralized CE, especially for users with highly intermittent internet connections.
Even users with more consistent connections may prefer to not have a CE client constantly scanning for connections to teammates (for example, in some battery-powered scenarios).

On the other hand, servers present security and privacy risks.
For example, attackers can monitor server communications to gather metadata about team communications, and censors can block servers more easily than they can block purely P2P communications.
Furthermore, assuming any kind of server in a system raises questions about who is responsible for operating and paying for it.

In UWS we find a compromise on the server issue by assuming that every user has a passive cloud storage location of some kind.
For many users, this will be an account with commodity services like Dropbox, Google Drive or Microsoft OneDrive.
More technically savvy users with stronger privacy concerns can run their own storage server at home or with a lower profile cloud provider.
In our protocol design we strive to minimize the API and performance expectations of the storage server in order to maximize the flexibility that users have with filling that role.
The interface is essentially just simple HTTP file upload/download.
The only slightly exotic support that is expected is something like the HTTP If-Match features.

Details of the storage server API and further server-related mitigations are given below.

(Users with yet higher privacy concerns might be able to do their storage as a Tor Onion Service, or similar.
Further investigation required.)

\subsection{Offline Editing}

Good support for users making concurrent edits while disconnected is necessary to match the usability of popular CE systems.
A system must both automatically merge concurrent edits when feasible and provide for conflict resolution otherwise.

Many recent CE systems have put a lot of focus on automatic merging with some flavor of operational transformations (OT) or conflict-free replicated data types (CRDTs).
These concurrent edit merging frameworks are useful, but do not themselves provide support for identification and resolution of conflicts at the level of application semantics.

We prefer to base our protocol's core data model on a Bayou-like totally ordered chain of edits/transactions.
It is conceptually straightforward to build OT or CRDT like data abstractions on top of such a model.
More details below.

\subsection{Remote Access}

Users should have reasonable remote access to their documents/database in the sense that they can log in from a new device and have the same usability experience (or nearly so) as from a computer that they use regularly.

An important implication of this principle is that a protocol must support reasonably efficient query from the storage server.
In other words, it is unacceptable to require users to download a team's complete database to a new device before they can start working with it.

\subsection{Latency}

Obviously, the lower the latency that a system can provide, the better.
Our current prototype uses cloud storage upload as the only communication medium, which imposes a fairly high minimum latency for communicating edits to teammates (multiple seconds is common).

It should be possible to use P2P connections for lower latency when teammates are simultaneously online, but we have not investigated this in any detail yet.
Several other projects have explored P2P CE, so the only question is how hard it is to hybridize these kinds of communication/storage systems with UWS.

Some applications should work fine, even with relatively high latency.
For example, a shared calendar or reservation system.

\subsection{Scalability}

\subsubsection{Team Size}

Team size scalability is perhaps the most important weakness of UWS in its current incarnation.
The edit ordering protocol uses vector clock timestamps, so edit size scales linearly with team size.
Also by default there is no hierarchical struture to teams, so everyone needs to monitor every other teammate's storage locaction for updates.

\subsubsection{Storage}

By default every teammate keeps a complete copy of the team's database in their storage location.
So the aggregate storage requirements grow linearly with team size.


\section{Concurrent Edits}

There has been a great deal of research on detecting conflicts between and automatically merging concurrent edits.
The UWS concurrent edit merging algorithm is closely related to the Bayou approach.

\subsection{Bayou}

In Bayou, the data shared between teammates is a linear chain of edits (\emph{Writes} in their jargon).
Of course, the chains at different clients cannot be identical at all times.
The Bayou system divides the chain into two sections.
The more recent edits are in the tentative section.
Edits in the tentative section might be reordered or superseded by as-yet-unseen edits.
A consistency protocol is run to decide when edits can graduate from the tentative section to the stable/committed section.
The order of edits in the stable section has been agreed upon, and cannot be changed by subsequent messages from teammates (unless the protocol is violated).

The strength of the Bayou approach is that it gives application programmers flexibility ...

\subsection{Updating Bayou}

UWS refines the Bayou model in a few ways:

\begin{itemize}
\item Dumber servers
\item More shades of grey between tentative and committed
\item More flexible querying of not-yet-stable edits
\end{itemize}

\subsubsection{Dumber Servers}

The Bayou project did not consider security/privacy as an issue at all; they were simply trying to make a good decentralized concurrent editing system.
Therefore, the designers did not consider the potential problems of malicious server operators or network monitors.
As a consequence, they assume servers will play a larger role in merging concurrent edits than we are willing to accept.

consequences?

\subsubsection{How Tentative is Tentative?}

As mentioned above, Bayou splits each client's edit chain 

\end{document}

tracking number

110516049

sharon

200lb
499 6285 vince

Carl Brown
Jeff Mcall
Drew Hutchinson
